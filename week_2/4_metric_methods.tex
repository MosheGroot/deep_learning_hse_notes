\documentclass{article}
\usepackage[T1,T2A]{fontenc}
\usepackage[utf8]{inputenc}
\usepackage[english,russian]{babel}

\usepackage[left=3cm,right=3cm,
    top=3cm,bottom=3cm,bindingoffset=0cm]{geometry}

\usepackage{graphicx}
\usepackage{color}
\usepackage{hyperref}
\usepackage{amsmath}
\usepackage{amsfonts}
\usepackage{amssymb}


\usepackage{setspace}
\usepackage{indentfirst}
\usepackage{textcomp}
\usepackage{ifthen}
\usepackage{calc}
\usepackage{mathrsfs}
\usepackage[dvipsnames]{xcolor}


\title{Введение в машинное обучение}
\author{Национальный исследовательский университет "Высшая школа экономики" \and Yandex School of Data Analysis\\\\
Неофициальный конспект по курсу.}

\begin{document}
\maketitle

\section{Метрические методы}

Следующие методы, которые мы рассмотрим, будут \textbf{метрические методы}, которые используют функции \textit{расстояния} или \textit{метрики} в пространстве объектов.

Исходная идея заключается в предположении, что в практических задачах часто встречаются зависимости, которые непрерывны (хотя это справедливо скорее для задач регрессии, уточнения описаны ниже):

\begin{itemize}
\item \textbf{Гипотеза непрерывности} (для регрессии):

\begin{center}
\textcolor{red}{близким объектам соответствуют близкие объекты.}
\end{center}

\item \textbf{Гипотеза компактности} (для классификации):

\begin{center}
\textcolor{red}{близкие объекты, как правило, лежат в одном классе.}
\end{center}

\end{itemize}

Выглядит логично, однако, как формализовать эту <<близость>>?

\begin{itemize}
\item \textbf{Формализация понятия <<близости>>:}

\qquad Задана функция расстояния $\rho : X \times X \rightarrow [0, \infty)$.
\end{itemize}

По сути эта функция от пары объектов, которая паре ставит соответствие — неотрицательное число.

Часто также накладывают требование, чтобы это была метрика в пространстве объектов, то есть чтобы она была и симметричной, и выполнялось неравенство треугольника. Однако формально в методах нигде не используется предположение о том, что это метрика, поэтому, в общем-то, и функции расстояния, не являющиеся метриками, нам тоже подходят.
\\

Самым известным примером расстояния, наверное, является евклидово расстояние в признаковом пространстве:

\begin{itemize}
\item \textbf{Пример.} Евклидово пространствов и его обобщение:

$$\rho(x, x_i) = \Biggr(\sum\limits_{j = 1}^n |x^j - x_i^j|^2\Biggl)^\frac{1}{2}
\qquad\qquad\qquad
\rho(x, x_i) = \Biggr(\textcolor{red}{w_j}
\sum\limits_{j = 1}^n |x^j - x_i^j|^{\textcolor{red}{p}}
\Biggl)^\frac{1}{\textcolor{red}{p}}$$

$x = (x^1, \ldots, x^n)$ --- вектор признаков объекта $x$.

$x_i = (x_i^1, \ldots, x_i^n)$ --- вектор признаков объекта $x_i$.

\end{itemize}

\end{document}
